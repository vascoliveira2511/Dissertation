\chapter*{Abstract}

Until recently, the business communication scenery was dominated by marketed Unified Communication (UC) solutions to the likes of Advanced Business Communication (ABC), a renowned product from Altice Labs, directed to enterprises, offering a myriad of Voice over IP (VoIP) services. Still, lately, there has been a shift toward using more collaboration-centric incorporated tools in businesses, specifically Microsoft Teams (MS Teams).

The notorious use of more collaboration-centered tools can be attributed to various factors, such as the shift to a more remote work lifestyle in the industry or the change in perspective of both employees and employers post-COVID-19.

This adoption does not mean the UC influence has lost its merit; on the contrary, it presents a new opportunity for creating Unified Communication and Collaboration (UC\&C) integrations. Therein lies the purpose of this proposal: integrating the ABC solution with MS Teams. With it, the intended result is a platform that offers all the functionalities provided by the UC solution, combined with the collaborative prowess of the other, obtaining a user-centric, easy-to-use, all-in-one solution enhancing business communication.

This project aims to explore, architect, and execute a functional prototype of the integration while focusing on ease of provision, associated costs, time-to-market of the solution, and the use of open-source software. Meanwhile, a comparative study of existing integrations and their methods is explored to determine the most appropriate one for the assignment. The ultimate aim of this solution is for it to be installed in a cloud environment using container technology for better scalability and manageability.

For the solution at hand, innovative aspects are considered, such as cloud-based architectures, Docker and Kubernetes, Software as a Service (SaaS), Session Initiation Protocol (SIP), and the Representational State Transfer (REST) Application Programming Interface (API).

\paragraph{Keywords} Altice Labs, VoIP, ABC, Microsoft Teams, Cloud Environment, Docker, Kubernetes, REST API, SaaS, SIP

\cleardoublepage

\chapter*{Resumo}

Até recentemente, o cenário da comunicação empresarial era dominado por soluções comercializadas de Comunicação Unificada (UC) como o Advanced Business Communication (ABC), um produto de renome da Altice Labs, dirigido a empresas, oferecendo uma infinidade de serviços de Voice over IP (VoIP). Ainda assim, ultimamente, tem havido uma mudança no sentido do uso de ferramentas mais centradas na colaboração nas empresas, especificamente no uso Microsoft Teams (MS Teams).

A notória utilização de ferramentas de colaboração pode ser atribuída a vários fatores, como a mudança para um estilo de vida de trabalho mais remoto na indústria ou a mudança de perspetiva tanto dos trabalhadores como dos empregadores pós-COVID-19.

Esta adoção não significa que a influência da UC tenha perdido o seu mérito, mas surge uma nova oportunidade para a criação de integrações de Comunicação e Colaboração Unificadas (UC\&C). É aí que reside o propósito desta proposta: a integração da solução ABC com MS Teams

Com esta integração, o resultado pretendido é uma plataforma que oferece todas as funcionalidades proporcionadas pela solução UC, aliadas à capacidade colaborativa da outra, obtendo uma solução tudo-em-um, centrada no utilizador e fácil de usar, melhorando a comunicação empresarial.

Este projeto visa explorar, arquitetar e executar um protótipo funcional da integração, concentrando-se na facilidade de fornecimento, nos custos associados, no tempo de colocação da solução no mercado e no uso de software de código aberto. Enquanto isso, um estudo comparativo das integrações existentes e os seus métodos é explorado para determinar o mais adequado para a tarefa. O objetivo final desta solução é ser instalada num ambiente de nuvem usando tecnologia de containers para melhor escalabilidade e capacidade de administração.

Para a solução em questão são considerados aspetos inovadores, como arquiteturas baseadas em nuvem, Docker e Kubernetes, Software as a Service (SaaS), Session Initiation Protocol (SIP) e Representational State Transfer (REST) Application Programming Interface (API).

\paragraph{Palavras-chave} Altice Labs, VoIP, ABC, Microsoft Teams, Cloud Environment, Docker, Kubernetes, REST API, SaaS, SIP

\cleardoublepage
