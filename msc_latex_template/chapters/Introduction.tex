\chapter{Introduction}

The following introductory section acts as a base for the ensuing chapters, outlining the context in which this project inserts itself, the motivation and objectives that lead to its idealization, and an explanation of the decided structure to guide the reader.  

\section{Context}

The term \gls{uc} is familiar in the technological industry, with history citing that its use began around the mid-1990s, although its complexity and contents are ever-evolving. By its name alone, one can assume that \gls{uc} searches for unifying communication technologies, traditional or novel, such as speech, text, and video, and associated devices like phones and computers \citep{Riemer2009}. With a single medium containing so many services, the advantages are clear to enterprises, such as improving business processes, reducing costs, and ultimately augmenting customer satisfaction \citep{Jiafeng2012}.

Altice Labs, a renowned corporation leading in innovation, launched its own \gls{uc} solution, mainly targeted toward businesses, named \gls{abc}. It is a solution for Service Providers seeking readily accessible and comprehensive business-level communication that, in part, they can offer to their corporate clients \citep{alticelabsUnifiedCommunications}. Their myriad of \gls{voip} services is not only procured nationally but also internationally, such as in the United States and the Dominican Republic.

After the events that transpired due to the COVID-19 pandemic, many businesses relied on this technology to communicate with employees and employers since in-person meetings or work could be dangerous and harmful to personnel \citep{10.1145/3498335}. However, a central part that needed to be explored thoroughly was its collaboration components; sure, one could make phone calls and even video calls, but the essential aspects, such as meetings, conference rooms, whiteboards, and calendars, needed to be more extensive and present.

Therefore, there was an enormous race to use more collaboration-centric platforms or integrate already existing \gls{uc} solutions into them, now known as \gls{ucc} solutions. They aim to reinforce these components while adapting or evolving the others simultaneously. With these simultaneous advances and events, a bubble was created; tools that once were less widely used now saw an explosion of use in the market. Applications such as Zoom and \gls{mst} saw their market share explode, with the latter noting 200 million daily meetings in April of 2020 \citep{10.1145/3498335}.

Now, the question of how solutions like \gls{abc} can compete with giants like \gls{mst} arises. However, the answer lies not in competition but in cooperation. Therein lies the true purpose of this proposal: to integrate \gls{abc} into \gls{mst}. With this integration, the unique backbone of \gls{uc} services offered by the former can be combined with the collaboration tools of the latter, creating an accurate and complete \gls{ucc} solution that will surely entice most corporations.

\section{Motivation and Objectives}

With the growing intent of businesses to incorporate \gls{ucc} solutions into their company via strategic plans and budget allocations, be it their own or from other providers \citep{Alias2018}, the competition in the sector has never been fiercer. Many big-name companies, such as the likes of RingCentral \citep{ringcentralAdvancedPhone} and 8x8 \citep{8x8VoiceMicrosoft}, have already taken the lead in integrating their \gls{uc} service into \gls{mst}.

Therefore, Altice Labs must capitalize on this opportunity and integrate its solution, \gls{abc}, into the market; from here, the proposition for the integration materializes. The suite of services offered by \gls{abc} will surely entice many corporations; their business model allows for new and legacy systems to be supported, which might alleviate some companies' burden in transitioning to a new system \citep{Alias2019}.

With this integration in mind, the following objectives/activities were designed to facilitate the success and completeness of the project:

\begin{itemize}
    \item Study and experimentation of the \gls{uc} services offered by \gls{abc};

    \item Comparative study of \gls{mst} with existing collaboration solutions in the market;

    \item Analysis of existing methods of integrations already in use with \gls{mst} by other \gls{uc} vendors;

    \item Selection of most adequate integration method via examination;

    \item Architecture design of the chosen solution;

    \item Application of Resilience and redundancy tests;

    \item Implementation of a functional prototype of the integration of \gls{abc} into \gls{mst};

    \item Installation in cloud environment via containers (Kubernetes).
    
\end{itemize}

With these delineated objectives, the proposal becomes well-structured and easy to manage.

%\section{Contributions and Results}

\section{Structure of the Dissertation}

Here follows the structure of this dissertation:

\begin{itemize}

    \item \textbf{Chapter 1: Introduction -} This chapter serves as a contextualization for the project, as well as providing the motivation and objectives behind it;

    \item \textbf{Chapter 2: State of the Art -} In this chapter, a guide is provided for all the technologies and research that envelop this assignment;

    \item \textbf{Chapter 3: The problem and its challenges -} This chapter focuses on issues and challenges that arise with the undertaking of this work as well as on how they are solved;

    \item \textbf{Chapter 4: Development -} In this chapter, the issuing work accomplished to ensure the completeness of the solution is documented in detail, detailing all the different steps executed.

    \item \textbf{Chapter 5: Testing and Analysis -} This chapter details all the tests accomplished to understand the tools at hand and the tests done on the final prototype of the proposed integration. An analysis of the obtained results is also presented to ensure that the results are as expected.

    \item \textbf{Chapter 6: Conclusions and future work -} In this final chapter, a conclusion is drawn from the job accomplished. A proposal for future work is also present.
    
\end{itemize}